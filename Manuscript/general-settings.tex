
% Commands for ensembles (serve il pacchetto amssymb)
\newcommand{\numberset}{\mathbb} 
\newcommand{\N}{\numberset{N}} 

% environment for systems
\newenvironment{sistema}%
  {\left\lbrace\begin{array}{@{}l@{}}}%
  {\end{array}\right.}

% definitions (requires amsthm)
\theoremstyle{definition} 
\newtheorem{definizione}{Definizione}

% teoremi, leggi e decreti (serve il pacchetto amsthm)
\theoremstyle{plain} 
\newtheorem{theorem}{Theorem}
\newtheorem{law}{Law}
%\newtheorem{decreto}[legge]{Decreto}
%\newtheorem{murphy}{Murphy}[section]


%*********************************************************************************
% Settings chapter title
%*********************************************************************************
\titleformat{\chapter}[hang]% "hang" instead of "block"
    {\normalfont\Large\sffamily}%
    {{\color{halfgray}\chapterNumber\thechapter%
    \hspace{10pt}\vline}  }{10pt}%
    {\spacedallcaps}

\newenvironment{unnumbered}%
{\setcounter{secnumdepth}{-1}}
{\setcounter{secnumdepth}{2}}

%*********************************************************************************
% Settings biblatex
%*********************************************************************************
%\defbibheading{bibliography}{%
%\cleardoublepage
%\manualmark
%\phantomsection 
%\addcontentsline{toc}{chapter}{\tocEntry{\bibname}}
%\chapter*{\bibname\markboth{\spacedlowsmallcaps{\bibname}}
%{\spacedlowsmallcaps{\bibname}}}}



%*********************************************************************************
% Settings listings
%*********************************************************************************
\lstset{language=[LaTeX]Tex,%C++,
    keywordstyle=\color{RoyalBlue},%\bfseries,
    basicstyle=\small\ttfamily,
    %identifierstyle=\color{NavyBlue},
    commentstyle=\color{Green}\ttfamily,
    stringstyle=\rmfamily,
    numbers=none,%left,%
    numberstyle=\scriptsize,%\tiny
    stepnumber=5,
    numbersep=8pt,
    showstringspaces=false,
    breaklines=true,
    frameround=ftff,
    frame=single
} 



%*********************************************************************************
% Settings hyperref
%*********************************************************************************
\hypersetup{%
    hyperfootnotes=false,pdfpagelabels,
    %draft,	% = elimina tutti i link (utile per stampe in bianco e nero)
    colorlinks=true, linktocpage=true, pdfstartpage=1, pdfstartview=FitV,%
    % decommenta la riga seguente per avere link in nero (per esempio per la stampa in bianco e nero)
    %colorlinks=false, linktocpage=false, pdfborder={0 0 0}, pdfstartpage=1, pdfstartview=FitV,% 
    breaklinks=true, pdfpagemode=UseNone, pageanchor=true, pdfpagemode=UseOutlines,%
    plainpages=false, bookmarksnumbered, bookmarksopen=true, bookmarksopenlevel=1,%
    hypertexnames=true, pdfhighlight=/O,%nesting=true,%frenchlinks,%
    urlcolor=webbrown, linkcolor=RoyalBlue, citecolor=webgreen, %pagecolor=RoyalBlue,%
    %urlcolor=Black, linkcolor=Black, citecolor=Black, %pagecolor=Black,%
    pdftitle={\myTitle},%
    pdfauthor={\textcopyright\ \myName, \myUni, \myFaculty},%
    pdfsubject={},%
    pdfkeywords={},%
    pdfcreator={pdfLaTeX},%
    pdfproducer={LaTeX with hyperref and ClassicThesis}%
}

%*********************************************************************************
% Margins settings optimized for  A4 format
%*********************************************************************************
\areaset[current]{336pt}{750pt}
\setlength{\marginparwidth}{7em}
\setlength{\marginparsep}{2em}%




%\hyphenation{Fortran ma-cro-istru-zio-ne nitro-idrossil-amminico}
\hyphenation{anisotropies an-i-so-trop-ies}
