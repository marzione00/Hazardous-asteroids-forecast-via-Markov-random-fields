Therefore if one sums the two equation, the following differential equation set will be obtained \cite{murray1999solar}:

\begin{equation}
m_{1}\ddot{\textbf{r}}_{1}+m_{2}\ddot{\textbf{r}}_{2}=0
\end{equation}

Its solution can be find with a direct integration \cite{murray1999solar}:

\begin{equation}
m_{1}\dot{\textbf{r}}_{1}+m_{2}\dot{\textbf{r}}_{2}=\textbf{a}
\end{equation}

\begin{equation}
m_{1}\textbf{r}_{1}+m_{2}\textbf{r}_{2}=\textbf{a}t+\textbf{b}
\end{equation}

Using the definition center of mass as \cite{murray1999solar}:

\begin{equation}
\textbf{R}=\frac{m_{1}\textbf{r}_{1}+m_{2}\textbf{r}_{2}}{m_{1}+m_{2}}
\end{equation}

the previous solution can be recast in the following form \cite{murray1999solar}:

\begin{equation}
\textbf{\dot{R}}=\frac{\textbf{a}}{m_{1}+m_{2}}
\end{equation}

\begin{equation}
\textbf{\dot{R}}=\frac{\textbf{a}t+\textbf{b}}{m_{1}+m_{2}}
\end{equation}

Where whe can see that 
